%\iffalse
\let\negmedspace\undefined
\let\negthickspace\undefined
\documentclass[journal,12pt,twocolumn]{IEEEtran}
\usepackage{cite}
\usepackage{amsmath,amssymb,amsfonts,amsthm}
\usepackage{algorithmic}
\usepackage{graphicx}
\usepackage{textcomp}
\usepackage{xcolor}
\usepackage{txfonts}
\usepackage{listings}
\usepackage{enumitem}
\usepackage{mathtools}
\usepackage{gensymb}
\usepackage{comment}
\usepackage[breaklinks=true]{hyperref}
\usepackage{tkz-euclide} 
\usepackage{listings}
\usepackage{gvv}                                        
\def\inputGnumericTable{}                                 
\usepackage[latin1]{inputenc}                                
\usepackage{color}                                            
\usepackage{array}                                            
\usepackage{longtable}                                       
\usepackage{calc}                                             
\usepackage{multirow}                                         
\usepackage{hhline}                                           
\usepackage{ifthen}                                           
\usepackage{lscape}

\newtheorem{theorem}{Theorem}[section]
\newtheorem{problem}{Problem}
\newtheorem{proposition}{Proposition}[section]
\newtheorem{lemma}{Lemma}[section]
\newtheorem{corollary}[theorem]{Corollary}
\newtheorem{example}{Example}[section]
\newtheorem{definition}[problem]{Definition}
\newcommand{\BEQA}{\begin{eqnarray}}
\newcommand{\EEQA}{\end{eqnarray}}
\newcommand{\define}{\stackrel{\triangle}{=}}
\theoremstyle{remark}
\newtheorem{rem}{Remark}

\begin{document}

\bibliographystyle{IEEEtran}
\vspace{3cm}
\title{\textbf{10.05.2}}
\author{EE23BTECH11053-R.Rahul$^{*}$% <-this % stops a space
}
\maketitle



\textbf{QUESTION:}\\
1. In the following APs, find the missing terms in the boxes:\\
(i) 2,, 26 \\
(ii) , 13, , 3\\
(iii) 5, , ,9\(\frac{1}{2}\)\\
(iv)'- 4', , , , , 6\\
(v) , 38, , , , '- 22'\\

Solution:\\
(i) $a_1$=2 $a_3$=26 $a_3$=a+2d\\
$\Longrightarrow$ 26=2+2*d $\Longrightarrow$ d=12\\
$a_2$=14\\
(ii) $a_2$=13 $a_4$=3 , $a_2$=a+d $a_4$=a+3d \\
$\Longrightarrow$ 3-13=2d $\Longrightarrow$=-5\\ $a_1$=18 ,$a_3$=8\\
(iii) $a_1$=5, $a_4$=9\(\frac{1}{2}\) $a_4$=a+3d\\
$\Longrightarrow$ 9\(\frac{1}{2}\)=5+3d ..3d=4\(\frac{1}{2}\) $\Longrightarrow$ d=1\(\frac{1}{2}\)\\
$a_2$=6\(\frac{1}{2}\) , $a_3$=8\\
(iv) $a_1$=-4 $a_6$=6 $a_6$=a+5d\\
$\Longrightarrow$ 6=-4+5d $\Longrightarrow$10=5d ... d=2\\
$a_2$=-2 $a_3$=0 $a_4$=2 $a_5$=4\\
(v)$a_2$=38 $a_6$=-22 \\
$\Longrightarrow$ -22-38=4d... d=-15\\
$a_1$=53 $a_3$=23 $a_4$=8 $a_5$=-7

\end{document}
